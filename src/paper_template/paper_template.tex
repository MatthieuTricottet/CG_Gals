\documentclass{aa}  
%
\usepackage{graphicx}
\usepackage[varg]{txfonts}
\usepackage{natbib,twoopt}
\usepackage[breaklinks=true]{hyperref} %% to avoid \citeads line fills
\hypersetup{
    colorlinks=true,
    linkcolor=blue,
    citecolor=blue,
    filecolor=magenta,      
    urlcolor=cyan,
    pdftitle={Overleaf Example},
    pdfpagemode=FullScreen,
    }

\usepackage[justification=centering]{caption}

\bibpunct{(}{)}{;}{a}{}{,}             %% natbib format for A&A and ApJ

\makeatletter
  \newcommandtwoopt{\citeads}[3][][]{\href{http://adsabs.harvard.edu/abs/#3}%
    {\def\hyper@linkstart##1##2{}%
     \let\hyper@linkend\@empty\citealp[#1][#2]{#3}}}
  \newcommandtwoopt{\citepads}[3][][]{\href{http://adsabs.harvard.edu/abs/#3}%
    {\def\hyper@linkstart##1##2{}%
     \let\hyper@linkend\@empty\citep[#1][#2]{#3}}}
  \newcommandtwoopt{\citetads}[3][][]{\href{http://adsabs.harvard.edu/abs/#3}%
    {\def\hyper@linkstart##1##2{}%
     \let\hyper@linkend\@empty\citet[#1][#2]{#3}}}
  \newcommandtwoopt{\citeyearads}[3][][]%
    {\href{http://adsabs.harvard.edu/abs/#3}
    {\def\hyper@linkstart##1##2{}%
     \let\hyper@linkend\@empty\citeyear[#1][#2]{#3}}}
\makeatother

% To add links in your PDF file, use the package "hyperref"
% with options according to your LaTeX or PDFLaTeX drivers.
%

\usepackage{graphicx}	% Including figure files
\usepackage{color}
\usepackage{array}
\usepackage{multirow}
\usepackage{makecell}
\usepackage{ae,aecompl}
\usepackage[utf8]{inputenc}
\usepackage{url}
% \usepackage{autonum} % only cited equations are numbered

\usepackage{booktabs}
\usepackage{amsmath}


\usepackage{listings}
\usepackage{xcolor}

\newcommand{\diazg}{D\'{\i}az-Gim\'enez}

\newcommand{\Quenched}{Quenched}
\newcommand{\Passive}{Passive}
\newcommand{\Starforming}{Star forming}


\newcommand{\CB}{Control\textsubscript{4B}} 
\newcommand{\CC}{Control\textsubscript{4C}} 
\newcommand{\CG}{CG\textsubscript{4}} 
\newcommand{\RG}{RG\textsubscript{4}} 
\newcommand{\kms}{\,\mathrm{km\,s}^{-1}}
\newcommand{\gap}{$\Delta R_{1,2}$}
\newcommand{\msun}{\mathrm{M}_\odot}

\usepackage{color}
\usepackage{array}
\usepackage{multirow}
\usepackage{makecell}
\usepackage{ae,aecompl}
\usepackage[utf8]{inputenc}
\usepackage{url}
% \usepackage{autonum} % only cited equations are numbered

\definecolor{darkgreen}{rgb}{0.0,0.5,0.0}
\definecolor{darkred}{rgb}{0.75,0.,0.2}
\definecolor{magenta}{rgb}{0.8,0,0.8}
\definecolor{purple}{rgb}{0.5,0,0.5}
\definecolor{gray}{rgb}{0.5,0.6,0.7}
\newcommand{\hide}[1]{\textcolor{gray}{#1}}
\newcommand{\red}[1]{\textcolor{red}{#1}}
\newcommand{\blue}[1]{\textcolor{blue}{#1}}
\newcommand\boldblue[1]{\textcolor{blue}{\mathbf{#1}}}
\newcommand\boldred[1]{\textcolor{darkred}{\mathbf{#1}}}
\newcommand{\gam}[1]{\textcolor{darkred}{\bf Gary: #1}}
\newcommand{\gm}[1]{\textcolor{darkred}{#1}}
\newcommand{\referee}[1]{\textcolor{red}{\bf Referee: #1}}
\newcommand{\MT}[1]{\textcolor{darkgreen}{\bf Matthieu: #1}}
\newcommand{\mt}[1]{\textcolor{darkgreen}{#1}}



\lstset{
  language=SQL,
  basicstyle=\ttfamily\small,
  keywordstyle=\color{blue}\bfseries,
  stringstyle=\color{red},
  commentstyle=\color{gray}\itshape,
  morekeywords={SpecObj,PhotoObj,galSpecExtra,galSpecLine},
  numbers=left,
  numberstyle=\tiny\color{gray},
  stepnumber=1,
  numbersep=5pt,
  showstringspaces=false,
  tabsize=4,
  breaklines=true,
  frame=single,
  captionpos=b
}

\graphicspath


\begin{document}
\nolinenumbers
\title{Are galaxies in compact groups special?}

   \author{Matthieu Tricottet\inst{1} \thanks{matthieu.tricottet@gmail.com}
           \and Gary A. Mamon\inst{2}
           \and Eugenia \diazg\inst{3,4}}

   \institute{  
                80, rue d'Al\'esia, 75014 Paris, France
                \and Institut d'Astrophysique de Paris (UMR 7095: CNRS \& Sorbonne Universit\'e), 98 bis boulevard Arago, 75014 Paris, France
                \and CONICET. Instituto de Astronom\'ia Te\'orica y Experimental (IATE), Laprida 854, X5000BGR, C\'ordoba, Argentina
                \and Universidad Nacional de C\'ordoba (UNC). Observatorio Astron\'omico de C\'ordoba (OAC), Laprida 854, X5000BGR, C\'ordoba, Argentina
            }

   \date{Received XXX; accepted YYY}

  \abstract{
    We investigate the properties of galaxies in compact groups (CGs) and compare them to a control sample of galaxies.  
   }

   \keywords{galaxies: clusters: general -- catalogs}

   \maketitle

\section{Introduction}\label{sec:intro}

\section{Data}
\subsection{Samples}
We use the compact groups and control samples we built in our previous article \cite{Tricottet+25}. 
We neververtheless briefly recall here how they were built. \mt{Explain.} 
Contrarily to our previous article, though, we remove CGs that we classified as \emph{split} following \cite{Zheng&Shen21}.
In this process, our initial compact group sample of <<CG4_Groups_withsplit_N>> groups reduced to 
<<CG4_Groups_nonsplit_N>>.

We use the SDSS DR <<DATA_RELEASE>> to retrieve masses, star formation rates (SFRs) and morphologies of galaxies 
assessed in Galaxy Zoo \cite{GalaxyZoo1}. We specifically extracted fields \texttt{sfr\_tot\_p50}, 
\texttt{specsfr\_tot\_p50} and \texttt{lgm\_tot\_p50} from the \texttt{galSpecExtra} table, and
\texttt{p\_el\_debiased} and \texttt{p\_cs\_debiased} from the \texttt{zooSpec} table.
       

\subsection{sSFR}
To build a general star formation classification that could be uniformly applied to all our samples, we extracted
 galaxies from the SDSS DR <<DATA_RELEASE>>  
through the query displayed in listing \ref{lst:query}. 
% The number of galaxies in each sample is shown in table \ref{tab:sample_sizes}.

\begin{lstlisting}[caption={Query used for selecting spectral \& photometric data from the SDSS},label={lst:query}]
    <<SDSS_query>>
\end{lstlisting}

We select non-AGN galaxies by first placing them on the classical BPT diagnostic of \cite{Veilleux87} and requiring measured values of both emission-line ratios
\[
\log_{10}\!\bigl([\mathrm{N\,II}]\lambda6584/\mathrm{H}\alpha\bigr)
\quad\text{and}\quad
\log_{10}\!\bigl([\mathrm{O\,III}]\lambda5007/\mathrm{H}\beta\bigr).
\]
A galaxy is flagged as an AGN if it satisfies either  
\[
\log_{10}\frac{[\mathrm{N\,II}]}{\mathrm{H}\alpha} > 0,
\]  
or if it lies above the empirical demarcation of \cite{Kauffmann+03a}:  
\[
\log_{10}\frac{[\mathrm{O\,III}]}{\mathrm{H}\beta}
> \frac{0.61}{\log_{10}([\mathrm{N\,II}]/\mathrm{H}\alpha)-0.05} \;+\; 1.3,
\]
in which case it is removed from the non-AGN sample.  All remaining galaxies—including those with missing line ratios—are retained as non-AGN.
The non-AGN selection process is shown on figure \ref{fig:BPT}.


\begin{figure}
    \centering
     \includegraphics[width=\columnwidth]{BPT_diagram.pdf}
     \caption{BPT diagram showing our separation between ordinary and AGN galaxies}
     \label{fig:BPT}
\end{figure}


We identify the galaxies having a \texttt{specsf\_tot\_p50} of -9999 in the SDSS as quenched 
(for graphic representation purposes, we give it here an arbitrary low value of $10^{<<sSFR_QUENCHED>>}$ Gyr\(^{-1}\)).
We then consider the other ones and sort them between the 
star-forming and passive populations using a two‐component Gaussian Mixture Model (GMM) in the \(\log M_*\)–\(\log\mathrm{sSFR}\) plane.
Its parameters determined by minimizing the Kullback–Leibler divergence between the empirical and model densities 
\cite{Kullback1951,McLachlanPeel2000}. For any trial parameter vector \(\boldsymbol{\theta}\), we reconstruct the mixture means 
\(\{\boldsymbol{\mu}_i\}\), covariances \(\{\mathbf{\Sigma}_i\}\), and weights \(\{w_i\}\) via a Cholesky‐like decomposition and 
logistic mapping, optionally constraining the second component’s mean to isolate the passive population \cite{Dempster1977}. 
The KL divergence is estimated by constructing a 2D histogram of the data and summing \(p_\mathrm{data}\ln(p_\mathrm{data}/p_\mathrm{GMM})\), 
with a small regularization \(\epsilon\) to avoid singularities \cite{Tojeiro2009,Bisigello2018}. We optimize \(\boldsymbol{\theta}\) 
via L-BFGS-B (with fallback to Nelder–Mead) over multiple guided and random initializations, selecting the solution with lowest divergence 
\cite{Pedregosa2011,Bishop2006}. Initial means are seeded by a median split in \(\log\mathrm{sSFR}\) to improve convergence and robustness 
against local minima \cite{Rousseeuw1987}. In the final fit, one component naturally aligns with the high-\(\mathrm{sSFR}\) “star‐forming 
sequence,” while the other captures the low-\(\mathrm{sSFR}\) “passive” cloud, providing a data‐driven division that agrees with previous 
multi‐Gaussian decompositions of the SFR–\(M_*\) plane \cite{Wuyts2011,Hahn2019}. The decision boundaries, defined as the 
loci where the posterior probabilities of adjacent components are equal, provide an objective criterion for 
delineating star‐forming from passive galaxies.

\mt{Add relevant GM parameters found by the algorithm.}

The sSFR status for each sample is shown in table \ref{tab:sSFR_status}. Tere is no significant differences of for \<<sSFR_status[2]>>{} proportion 
between \CG{} and the control samples (probabilities estimated using Fisher exact test are  
<<pval_Control4B_Starforming_vs_CG>>  versus \CB{} sample, 
 <<pval_Control4C_Starforming_vs_CG>>  versus \CC{} and  <<pval_RG4_Starforming_vs_CG>>  versus \RG).

 Restrictions of the comparison to BGGs and satellites only are also shown in table \ref{tab:sSFR_status}. 
 The p-values found from <<BGG_sSFR_tests>> are 
  <<BGG_star_forming_pvalue_Control4B_vs_CG4>> against \CB,
  <<BGG_star_forming_pvalue_Control4C_vs_CG4>> against \CC{} and 
  <<BGG_star_forming_pvalue_RG4_vs_CG4>>  against \RG. 
There is thus no significant difference in the fraction of star-forming BGGs nor satellites between \CG{} and our control samples either.

% we find <<CG4_BGG_N_Starforming>> star-forming BGGs (<<CG4_BGG_fracpc_Starforming>>\% of the BGGs),
% <<CG4_BGG_N_Passive>> Passive BGGs (<<CG4_BGG_fracpc_Passive>>\%) and <<CG4_BGG_N_Quenched>> Quenched BGGs (<<CG4_BGG_fracpc_Quenched>>\%)
%  in \CG{}.
%   This is to be compared to 
%  <<Control4B_BGG_N_Starforming>> (<<Control4B_BGG_fracpc_Starforming>>\%) Starforming, 
%  <<Control4B_BGG_N_Passive>> (<<Control4B_BGG_fracpc_Passive>>\%) Passive and
%  <<Control4B_BGG_N_Quenched>> (<<Control4B_BGG_fracpc_Quenched>>\%) in \CB{}, 
%  <<Control4C_BGG_N_Starforming>> (<<Control4C_BGG_fracpc_Starforming>>\%) Starforming,
%  <<Control4C_BGG_N_Passive>> (<<Control4C_BGG_fracpc_Passive>>\%) Passive and
%  <<Control4C_BGG_N_Quenched>> (<<Control4C_BGG_fracpc_Quenched>>\%)  Quenched in \CC{},
%   and <<RG4_BGG_N_Starforming>> (<<RG4_BGG_fracpc_Starforming>>\%) Starforming,
%   <<RG4_BGG_N_Passive>> (<<RG4_BGG_fracpc_Passive>>\%) Passive and
%   <<RG4_BGG_N_Quenched>> (<<RG4_BGG_fracpc_Quenched>>\%) Quenched
%   in \RG{}. 
  
\begin{table*}
    \centering
    \begin{tabular}{llccc}
        \toprule
            \textbf{Sample} & & \textbf{  <<sSFR_status[0]>>  } & \textbf{  <<sSFR_status[1]>>  } & \textbf{  <<sSFR_status[2]>>  } \\
            \CG & &  <<CG4_Gals_NQuenched>>  (<<CG4_Gals_NQuenched_pc>> \%) &  <<CG4_Gals_NPassive>>  (<<CG4_Gals_NPassive_pc>> \%)&  <<CG4_Gals_NStarforming>>  (<<CG4_Gals_NStarforming_pc>> \%) \\
             & \emph{BGG} &  \emph{<<CG4_BGG_NQuenched>>  (<<CG4_BGG_NQuenched_pc>> \%)} &  \emph{<<CG4_BGG_NPassive>>  (<<CG4_BGG_NPassive_pc>> \%)} &  \emph{<<CG4_BGG_NStarforming>>  (<<CG4_BGG_NStarforming_pc>> \%)} \\
             & \emph{Satellites} &  \emph{<<CG4_Sat_NQuenched>>  (<<CG4_Sat_NQuenched_pc>> \%)} &  \emph{<<CG4_Sat_NPassive>>  (<<CG4_Sat_NPassive_pc>> \%)} &  \emph{<<CG4_Sat_NStarforming>>  (<<CG4_Sat_NStarforming_pc>> \%)} \\
            \CB & &  <<Control4B_Gals_NQuenched>>  (<<Control4B_Gals_NQuenched_pc>> \%) &  <<Control4B_Gals_NPassive>>  (<<Control4B_Gals_NPassive_pc>> \%) &  <<Control4B_Gals_NStarforming>>  (<<Control4B_Gals_NStarforming_pc>> \%) \\
             & \emph{BGG} &  \emph{<<Control4B_BGG_NQuenched>>  (<<Control4B_BGG_NQuenched_pc>> \%)} &  \emph{<<Control4B_BGG_NPassive>>  (<<Control4B_BGG_NPassive_pc>> \%)} &  \emph{<<Control4B_BGG_NStarforming>>  (<<Control4B_BGG_NStarforming_pc>> \%)} \\
             & \emph{Satellites} &  \emph{<<Control4B_Sat_NQuenched>>  (<<Control4B_Sat_NQuenched_pc>> \%)} &  \emph{<<Control4B_Sat_NPassive>>  (<<Control4B_Sat_NPassive_pc>> \%)} &  \emph{<<Control4B_Sat_NStarforming>>  (<<Control4B_Sat_NStarforming_pc>> \%)} \\
            \CC & &  <<Control4C_Gals_NQuenched>>  (<<Control4C_Gals_NQuenched_pc>> \%) &  <<Control4C_Gals_NPassive>>  (<<Control4C_Gals_NPassive_pc>> \%) &  <<Control4C_Gals_NStarforming>>  (<<Control4C_Gals_NStarforming_pc>> \%) \\
             & \emph{BGG} &  \emph{<<Control4C_BGG_NQuenched>>  (<<Control4C_BGG_NQuenched_pc>> \%)} &  \emph{<<Control4C_BGG_NPassive>>  (<<Control4C_BGG_NPassive_pc>> \%)} &  \emph{<<Control4C_BGG_NStarforming>>  (<<Control4C_BGG_NStarforming_pc>> \%)} \\
             & \emph{Satellites} &  \emph{<<Control4C_Sat_NQuenched>>  (<<Control4C_Sat_NQuenched_pc>> \%)} &  \emph{<<Control4C_Sat_NPassive>>  (<<Control4C_Sat_NPassive_pc>> \%)} &  \emph{<<Control4C_Sat_NStarforming>>  (<<Control4C_Sat_NStarforming_pc>> \%)} \\
            \RG & &  <<RG4_Gals_NQuenched>>  (<<RG4_Gals_NQuenched_pc>> \%) &  <<RG4_Gals_NPassive>>  (<<RG4_Gals_NPassive_pc>> \%) &  <<RG4_Gals_NStarforming>>  (<<RG4_Gals_NStarforming_pc>> \%)\\
            & \emph{BGG} &  \emph{<<RG4_BGG_NQuenched>>  (<<RG4_BGG_NQuenched_pc>> \%)} &  \emph{<<RG4_BGG_NPassive>>  (<<RG4_BGG_NPassive_pc>> \%)} &  \emph{<<RG4_BGG_NStarforming>>  (<<RG4_BGG_NStarforming_pc>> \%)}\\
            & \emph{Satellites} &  \emph{<<RG4_Sat_NQuenched>>  (<<RG4_Sat_NQuenched_pc>> \%)} &  \emph{<<RG4_Sat_NPassive>>  (<<RG4_Sat_NPassive_pc>> \%)} &  \emph{<<RG4_Sat_NStarforming>>  (<<RG4_Sat_NStarforming_pc>> \%)}\\
         \midrule
            \emph{SDSS selection} & &  \emph{<<SDSS_NQuenched>> (<<SDSS_NQuenched_pc>> \%)} &  \emph{<<SDSS_NPassive>>  (<<SDSS_NPassive_pc>> \%)} &  \emph{<<SDSS_NStarforming>>  (<<SDSS_NStarforming_pc>> \%)}\\
         \midrule
        \bottomrule
    \end{tabular}
    \caption{Number of galaxies in each sSFR status for each sample.}
    \label{tab:sSFR_status}
\end{table*}

        
    

\begin{figure}
    \centering
     \includegraphics[width=\columnwidth]{density_original_vs_GMMfit.pdf}
     \caption{}
     \label{fig:density_original_vs_GMMfit}
\end{figure}

\begin{figure}
    \centering
     \includegraphics[width=\columnwidth]{sSFR_classification.pdf}
     \caption{sSFR vs mass, star-forming/passive limit and Zoo morphologies for the SDSS control sample, 
     excluding quenched galaxies.}
     \label{fig:sSFR_classification}
\end{figure}

\begin{figure}
    \centering
     \includegraphics[width=\columnwidth]{residual_sSFR_histogram.pdf}
     \caption{}
     \label{fig:Residual_sSFR}
\end{figure}



\subsection{Morphology}
Morphologies are determined via the Galaxy Zoo citizen‐science decision tree, in which each galaxy image receives multiple 
independent volunteer classifications that are aggregated into debiaised vote fractions \cite{GalaxyZoo1}. Those fractions are provided in the 
\texttt{zooSpecz} table of the SDSS databasethrough the \texttt{p\_el\_debiased} and \texttt{p\_cs\_debiased} columns. We attribute to galaxies the
morphology ``elliptical'' if the \texttt{p\_el\_debiased} value is greater than 0.5, and ``spiral'' if the \texttt{p\_cs\_debiased} value is greater than 0.5. 
Other cases are considered as ``uncertain''. The morphologies found for each sample are shown in table \ref{tab:morphologies}.
Barnard two-sided exact tests were performed to compare the fraction of each morphology between \CG{} and each control sample. The p-values found are 
<<pval_Control4B_Elliptical_vs_CG_pc>> for ellipticals and <<pval_Control4B_Spiral_vs_CG_pc>> for spirals versus \CB{} sample, 
 <<pval_Control4C_Elliptical_vs_CG_pc>>  and <<pval_Control4C_Spiral_vs_CG_pc>>  versus \CC{} and  <<pval_RG4_Elliptical_vs_CG_pc>>  
 and <<pval_RG4_Spiral_vs_CG_pc>>  versus \RG. Morphologies of \CG, wich we identified as cores of richer groups in 
 \cite{Tricottet+25}, have morphologies significantly different from ordinary groups of four members. Figure \ref{fig:galaxies_morph_sfr} displays the sSFR against masses for galaxies from the SDSS selection and for \CG{}s galaxies.


\begin{table*}
    \centering
    \begin{tabular}{lccc}
        \toprule
            \textbf{Sample} & \textbf{  <<Morphologies[0]>>  } & \textbf{  <<Morphologies[1]>>  } & \textbf{  <<Morphologies[2]>>  } \\
            \CG &  <<CG4_Gals_N_Elliptical>>  (<<CG4_Gals_fraction_Elliptical_pc>> \%) &  <<CG4_Gals_N_Spiral>>  (<<CG4_Gals_fraction_Spiral_pc>> \%)&  <<CG4_Gals_N_Uncertain>>  (<<CG4_Gals_fraction_Uncertain_pc>> \%) \\
            \CB &  <<Control4B_Gals_N_Elliptical>>  (<<Control4B_Gals_fraction_Elliptical_pc>> \%) &  <<Control4B_Gals_N_Spiral>>  (<<Control4B_Gals_fraction_Spiral_pc>> \%) &  <<Control4B_Gals_N_Uncertain>>  (<<Control4B_Gals_fraction_Uncertain_pc>> \%) \\
            \CC &  <<Control4C_Gals_N_Elliptical>>  (<<Control4C_Gals_fraction_Elliptical_pc>> \%) &  <<Control4C_Gals_N_Spiral>>  (<<Control4C_Gals_fraction_Spiral_pc>> \%) &  <<Control4C_Gals_N_Uncertain>>  (<<Control4C_Gals_fraction_Uncertain_pc>> \%) \\
            \RG &  <<RG4_Gals_N_Elliptical>>  (<<RG4_Gals_fraction_Elliptical_pc>> \%) &  <<RG4_Gals_N_Spiral>>  (<<RG4_Gals_fraction_Spiral_pc>> \%) &  <<RG4_Gals_N_Uncertain>>  (<<RG4_Gals_fraction_Uncertain_pc>> \%)\\
            \emph{SDSS selection} &  \emph{<<SDSS_N_Elliptical>> (<<SDSS_fraction_Elliptical_pc>> \%)} &  \emph{<<SDSS_N_Spiral>>  (<<SDSS_fraction_Spiral_pc>> \%)} &  \emph{<<SDSS_N_Uncertain>>  (<<SDSS_fraction_Uncertain_pc>> \%)}\\
         \midrule
        \bottomrule
    \end{tabular}
    \caption{Number of galaxies of each morphology for each sample.}
    \label{tab:morphologies}
\end{table*}


\begin{figure}
    \centering
     \includegraphics[width=\columnwidth]{galaxies_sfr.pdf}
     \caption{sSFR and morphologies of galaxies in \CG{} and our SDSS selection. 
     Quenched galaxies are represented at an arbitrary low value of $10^{<<sSFR_QUENCHED>>}$ Gyr\(^{-1}\).}
     \label{fig:galaxies_morph_sfr}
\end{figure}

\subsection{Morphology vs sSFR}



Starforming \CG{} are composed of <<CG_Nb_Spiral_Starforming>> of their galaxies classified as spiral 
(<<CG_fracpc_Spiral_Starforming>>\% once galaxies having uncertain morphology removed) and 
<<CG_Nb_Elliptical_Starforming>> (<<CG_fracpc_Elliptical_Starforming>>\% once galaxies having uncertain morphology removed) elliptical.
These figures are <<Control4B_Nb_Spiral_Starforming>> (<<Control4B_fracpc_Spiral_Starforming>>\%) spiral
and <<Control4B_Nb_Elliptical_Starforming>> (<<Control4B_fracpc_Elliptical_Starforming>>\%) elliptical galaxies for \CB{}, 
<<Control4C_Nb_Spiral_Starforming>> (<<Control4C_fracpc_Spiral_Starforming>>\%) spiral
and <<Control4C_Nb_Elliptical_Starforming>> (<<Control4C_fracpc_Elliptical_Starforming>>\%) elliptical galaxies for \CC{}
and <<RG4_Nb_Spiral_Starforming>> (<<RG4_fracpc_Spiral_Starforming>>\%) spiral
and <<RG4_Nb_Elliptical_Starforming>> (<<RG4_fracpc_Elliptical_Starforming>>\%) elliptical galaxies for \RG{}. 
The probability that those fractions come from 
the same distribution is estimated from the <<Morph_sSFR_test>> as <<pval_Control4B_Starforming_vs_CG_pc>> against \CB,
<<pval_Control4C_Starforming_vs_CG_pc>> against \CC{} and <<pval_RG4_Starforming_vs_CG_pc>> against \RG. There is thus 
neither significant excess nor lack of 
starforming elliptical galaxies in \CG{} compared to all our control samples.

If we restrict the comparison to BGGs only, we find <<CG4_BGG_N_Elliptical>> Elliptical BGGs (<<CG4_BGG_fracpc_Elliptical>>\% of the BGGs) and
<<CG4_BGG_N_Spiral>> Spiral BGGs (<<CG4_BGG_fracpc_Spiral>>\%) 
 in \CG{}.
  This is to be compared to 
 <<Control4B_BGG_N_Elliptical>> (<<Control4B_BGG_fracpc_Elliptical>>\%) Elliptical and
 <<Control4B_BGG_N_Spiral>> (<<Control4B_BGG_fracpc_Spiral>>\%) Spiral galaxies in \CB{}, 
 <<Control4C_BGG_N_Elliptical>> (<<Control4C_BGG_fracpc_Elliptical>>\%) Elliptical and
 <<Control4C_BGG_N_Spiral>> (<<Control4C_BGG_fracpc_Spiral>>\%)  Spiral in \CC{},
  and  <<RG4_BGG_N_Elliptical>> (<<RG4_BGG_fracpc_Elliptical>>\%) Elliptical and
  <<RG4_BGG_N_Spiral>> (<<RG4_BGG_fracpc_Spiral>>\%) Quenched
  in \RG{}. 
The p-values found from <<BGG_morph_tests>> are 
  <<BGG_morph_pvalue_Control4B_vs_CG4>> against \CB,
  <<BGG_morph_pvalue_Control4C_vs_CG4>> against \CC{} and 
  <<BGG_morph_pvalue_RG4_vs_CG4>>  against \RG. 
There is thus no significant difference in the fraction of star-forming BGGs between \CG{} and our control samples.  

% \subsection{Dominance}
% We split groups into two categories: the dominated, for which the BGG has a luminosity fraction greater than the median of \RG.

\begin{appendix} %First appendix
\section*{Interpolation points for sSFR classification}

% \begin{table}[t]
% \centering
% \caption{Interpolation points for sSFR classification curve}
% \begin{tabular}{rr}
% \toprule
% $x$ & $y$ \\
% \midrule
% 
% 
% {{ "%.6g"|format(p.x) }} & {{ "%.6g"|format(p.y) }} \\
% 
% \\
% 
% \multicolumn{2}{c}{\rule{0pt}{0pt}}\\
% \multicolumn{2}{c}{No interpolation points available} \\
% 
% \bottomrule
% \end{tabular}
% \end{table}





\end{appendix}

\begin{acknowledgements}
We thank ... 
\end{acknowledgements}

\bibliographystyle{aa}
\bibliography{<<BIB_FILE>>}

\end{document}