\documentclass{aa}  
%
\usepackage{graphicx}
\usepackage[varg]{txfonts}
\usepackage{natbib,twoopt}
\usepackage[breaklinks=true]{hyperref} %% to avoid \citeads line fills
\hypersetup{
    colorlinks=true,
    linkcolor=blue,
    citecolor=blue,
    filecolor=magenta,      
    urlcolor=cyan,
    pdftitle={Overleaf Example},
    pdfpagemode=FullScreen,
    }
\bibpunct{(}{)}{;}{a}{}{,}             %% natbib format for A&A and ApJ

\makeatletter
  \newcommandtwoopt{\citeads}[3][][]{\href{http://adsabs.harvard.edu/abs/#3}%
    {\def\hyper@linkstart##1##2{}%
     \let\hyper@linkend\@empty\citealp[#1][#2]{#3}}}
  \newcommandtwoopt{\citepads}[3][][]{\href{http://adsabs.harvard.edu/abs/#3}%
    {\def\hyper@linkstart##1##2{}%
     \let\hyper@linkend\@empty\citep[#1][#2]{#3}}}
  \newcommandtwoopt{\citetads}[3][][]{\href{http://adsabs.harvard.edu/abs/#3}%
    {\def\hyper@linkstart##1##2{}%
     \let\hyper@linkend\@empty\citet[#1][#2]{#3}}}
  \newcommandtwoopt{\citeyearads}[3][][]%
    {\href{http://adsabs.harvard.edu/abs/#3}
    {\def\hyper@linkstart##1##2{}%
     \let\hyper@linkend\@empty\citeyear[#1][#2]{#3}}}
\makeatother

% To add links in your PDF file, use the package "hyperref"
% with options according to your LaTeX or PDFLaTeX drivers.
%

\usepackage{graphicx}	% Including figure files
\usepackage{color}
\usepackage{array}
\usepackage{multirow}
\usepackage{makecell}
\usepackage{ae,aecompl}
\usepackage[utf8]{inputenc}
\usepackage{url}
% \usepackage{autonum} % only cited equations are numbered

\usepackage{booktabs}
\usepackage{amsmath}


\usepackage{listings}
\usepackage{xcolor}

\newcommand{\diazg}{D\'{\i}az-Gim\'enez}

\newcommand{\Quenched}{Quenched}
\newcommand{\Passive}{Passive}
\newcommand{\Starforming}{Star forming}


\newcommand{\CB}{Control\textsubscript{4B}} 
\newcommand{\CC}{Control\textsubscript{4C}} 
\newcommand{\CG}{CG\textsubscript{4}} 
\newcommand{\RG}{RG\textsubscript{4}} 
\newcommand{\kms}{\,\mathrm{km\,s}^{-1}}
\newcommand{\gap}{$\Delta R_{1,2}$}
\newcommand{\msun}{\mathrm{M}_\odot}

\usepackage{color}
\usepackage{array}
\usepackage{multirow}
\usepackage{makecell}
\usepackage{ae,aecompl}
\usepackage[utf8]{inputenc}
\usepackage{url}
% \usepackage{autonum} % only cited equations are numbered

\definecolor{darkgreen}{rgb}{0.0,0.5,0.0}
\definecolor{darkred}{rgb}{0.75,0.,0.2}
\definecolor{magenta}{rgb}{0.8,0,0.8}
\definecolor{purple}{rgb}{0.5,0,0.5}
\definecolor{gray}{rgb}{0.5,0.6,0.7}
\newcommand{\hide}[1]{\textcolor{gray}{#1}}
\newcommand{\red}[1]{\textcolor{red}{#1}}
\newcommand{\blue}[1]{\textcolor{blue}{#1}}
\newcommand\boldblue[1]{\textcolor{blue}{\mathbf{#1}}}
\newcommand\boldred[1]{\textcolor{darkred}{\mathbf{#1}}}
\newcommand{\gam}[1]{\textcolor{darkred}{\bf Gary: #1}}
\newcommand{\gm}[1]{\textcolor{darkred}{#1}}
\newcommand{\referee}[1]{\textcolor{red}{\bf Referee: #1}}
\newcommand{\MT}[1]{\textcolor{darkgreen}{\bf Matthieu: #1}}
\newcommand{\mt}[1]{\textcolor{darkgreen}{#1}}



\lstset{
  language=SQL,
  basicstyle=\ttfamily\small,
  keywordstyle=\color{blue}\bfseries,
  stringstyle=\color{red},
  commentstyle=\color{gray}\itshape,
  morekeywords={SpecObj,PhotoObj,galSpecExtra,galSpecLine},
  numbers=left,
  numberstyle=\tiny\color{gray},
  stepnumber=1,
  numbersep=5pt,
  showstringspaces=false,
  tabsize=4,
  breaklines=true,
  frame=single,
  captionpos=b
}

\graphicspath{{figures/}}


\begin{document}
\nolinenumbers
\title{Are galaxies in compact groups special?}

   \author{Matthieu Tricottet\inst{1} \thanks{matthieu.tricottet@gmail.com}
           \and Gary A. Mamon\inst{2}
           \and Eugenia \diazg\inst{3,4}}

   \institute{  
                80, rue d'Al\'esia, 75014 Paris, France
                \and Institut d'Astrophysique de Paris (UMR 7095: CNRS \& Sorbonne Universit\'e), 98 bis boulevard Arago, 75014 Paris, France
                \and CONICET. Instituto de Astronom\'ia Te\'orica y Experimental (IATE), Laprida 854, X5000BGR, C\'ordoba, Argentina
                \and Universidad Nacional de C\'ordoba (UNC). Observatorio Astron\'omico de C\'ordoba (OAC), Laprida 854, X5000BGR, C\'ordoba, Argentina
            }

   \date{Received XXX; accepted YYY}

  \abstract{
    We investigate the properties of galaxies in compact groups (CGs) and compare them to a control sample of galaxies.  
   }

   \keywords{galaxies: clusters: general -- catalogs}

   \maketitle

\section{Introduction}\label{sec:intro}

\section{Data}
\subsection{Samples}
We use the compact groups and control samples we built in our previous article \cite{Tricottet+25}. 
We neververtheless briefly recall here how they were built. \mt{Explain.} 
Contrarily to our previous article, though, we remove CGs that we classified as \emph{split} following \cite{Zheng&Shen21}.
In this process, our initial compact group sample of 78 groups reduced to 
62.

We use the SDSS DR 16 to retrieve masses, star formation rates (SFRs) and morphologies of galaxies 
assessed in Galaxy Zoo \cite{GalaxyZoo1}. We specifically extracted fields \texttt{sfr\_tot\_p50}, 
\texttt{specsfr\_tot\_p50} and \texttt{lgm\_tot\_p50} from the \texttt{galSpecExtra} table, and
\texttt{p\_el\_debiased} and \texttt{p\_cs\_debiased} from the \texttt{zooSpec} table.
       

\subsection{sSFR}
To build a general star formation classification that could be uniformly applied to all our samples, we extracted
 galaxies from the SDSS DR 16  
through the query displayed in listing \ref{lst:query}. 
% The number of galaxies in each sample is shown in table \ref{tab:sample_sizes}.

\begin{lstlisting}[caption={Query used for selecting spectral \& photometric data from the SDSS},label={lst:query}]
    
    SELECT 
        s.specObjID,
        s.z,
        p.petroMag_r,
        p.objID,
        g.sfr_tot_p50, g.specsfr_tot_p50, g.lgm_tot_p50,
        l.h_alpha_eqw, l.h_beta_eqw, l.oiii_5007_eqw, l.nii_6584_eqw,
        l.h_alpha_flux, l.h_beta_flux, l.oiii_5007_flux, l.nii_6584_flux,
        z. p_el_debiased AS p_E,
        z. p_cs_debiased     AS p_S
    FROM SpecObj AS s
        JOIN PhotoObj AS p ON s.bestObjID = p.objID
        JOIN galSpecExtra as g ON s.specObjID = g.specObjID
        JOIN galSpecLine as l ON s.specObjID = l.specObjID
        JOIN zooSpec AS z ON s.specObjID = z.specObjID
    WHERE s.z BETWEEN 0.005 AND 0.0452
        AND (p.petroMag_r - p.extinction_r <= 17.77)
        AND s.class = 'GALAXY'
        AND g.lgm_tot_p50 > -1000
    
\end{lstlisting}

We select non-AGN galaxies by first placing them on the classical BPT diagnostic of \cite{Veilleux87} and requiring measured values of both emission-line ratios
\[
\log_{10}\!\bigl([\mathrm{N\,II}]\lambda6584/\mathrm{H}\alpha\bigr)
\quad\text{and}\quad
\log_{10}\!\bigl([\mathrm{O\,III}]\lambda5007/\mathrm{H}\beta\bigr).
\]
A galaxy is flagged as an AGN if it satisfies either  
\[
\log_{10}\frac{[\mathrm{N\,II}]}{\mathrm{H}\alpha} > 0,
\]  
or if it lies above the empirical demarcation of \cite{Kauffmann+03a}:  
\[
\log_{10}\frac{[\mathrm{O\,III}]}{\mathrm{H}\beta}
> \frac{0.61}{\log_{10}([\mathrm{N\,II}]/\mathrm{H}\alpha)-0.05} \;+\; 1.3,
\]
in which case it is removed from the non-AGN sample.  All remaining galaxies—including those with missing line ratios—are retained as non-AGN.
The non-AGN selection process is shown on figure \ref{fig:BPT}.


\begin{figure}
    \centering
     \includegraphics[width=\columnwidth]{BPT_diagram.pdf}
     \caption{BPT diagram showing our separation between ordinary and AGN galaxies}
     \label{fig:BPT}
\end{figure}


We identify the galaxies having a \texttt{specsf\_tot\_p50} of -9999 in the SDSS as quenched 
(for graphic representation purposes, we give it here an arbitrary low value of $10^{-16.0}$ Gyr\(^{-1}\)).
We then consider the other ones and sort them between the 
star-forming and passive populations using a two‐component Gaussian Mixture Model (GMM) in the \(\log M_*\)–\(\log\mathrm{sSFR}\) plane.
Its parameters determined by minimizing the Kullback–Leibler divergence between the empirical and model densities 
\cite{Kullback1951,McLachlanPeel2000}. For any trial parameter vector \(\boldsymbol{\theta}\), we reconstruct the mixture means 
\(\{\boldsymbol{\mu}_i\}\), covariances \(\{\mathbf{\Sigma}_i\}\), and weights \(\{w_i\}\) via a Cholesky‐like decomposition and 
logistic mapping, optionally constraining the second component’s mean to isolate the passive population \cite{Dempster1977}. 
The KL divergence is estimated by constructing a 2D histogram of the data and summing \(p_\mathrm{data}\ln(p_\mathrm{data}/p_\mathrm{GMM})\), 
with a small regularization \(\epsilon\) to avoid singularities \cite{Tojeiro2009,Bisigello2018}. We optimize \(\boldsymbol{\theta}\) 
via L-BFGS-B (with fallback to Nelder–Mead) over multiple guided and random initializations, selecting the solution with lowest divergence 
\cite{Pedregosa2011,Bishop2006}. Initial means are seeded by a median split in \(\log\mathrm{sSFR}\) to improve convergence and robustness 
against local minima \cite{Rousseeuw1987}. In the final fit, one component naturally aligns with the high-\(\mathrm{sSFR}\) “star‐forming 
sequence,” while the other captures the low-\(\mathrm{sSFR}\) “passive” cloud, providing a data‐driven division that agrees with previous 
multi‐Gaussian decompositions of the SFR–\(M_*\) plane \cite{Wuyts2011,Hahn2019}. The decision boundaries, defined as the 
loci where the posterior probabilities of adjacent components are equal, provide an objective criterion for 
delineating star‐forming from passive galaxies.

\mt{Add relevant GM parameters found by the algorithm.}

The sSFR status for each sample is shown in table \ref{tab:sSFR_status}. The probability for \Starforming{} proportion 
between \CG{} and each control sample to actually come from the same underlying distribution are, estimated using Fisher exact test, 
$4.8 \times 10^{-1}$  versus \CB{} sample, 
 $2.2 \times 10^{-1}$  versus \CC{} and  $4.1 \times 10^{-3}$  versus \RG.

\begin{table}
    \centering
    \begin{tabular}{lccc}
        \toprule
            \textbf{Sample} & \textbf{  Quenched  } & \textbf{  Passive  } & \textbf{  Starforming  } \\
            \CG &  4  (1.0 \%) &  128  (31.8 \%)&  270  (67.2 \%) \\
            \CB &  32  (0.8 \%) &  1400  (36.3 \%) &  2427  (62.9 \%) \\
            \CC &  31  (0.7 \%) &  1839  (39.1 \%) &  2839  (60.3 \%) \\
            \RG &  0  (0.0 \%) &  149  (55.6 \%) &  119  (44.4 \%)\\
            \emph{SDSS selection} &  \emph{634 (1.2 \%)} &  \emph{44192  (84.1 \%)} &  \emph{7705  (14.7 \%)}\\
         \midrule
        \bottomrule
    \end{tabular}
    \caption{Number of galaxies in each sSFR status for each sample.}
    \label{tab:sSFR_status}
\end{table}

        
    

\begin{figure}
    \centering
     \includegraphics[width=\columnwidth]{density_original_vs_GMMfit.pdf}
     \caption{}
     \label{fig:density_original_vs_GMMfit}
\end{figure}

\begin{figure}
    \centering
     \includegraphics[width=\columnwidth]{sSFR_classification.pdf}
     \caption{sSFR vs mass, star-forming/passive limit and Zoo morphologies for the SDSS control sample, 
     excluding quenched galaxies.}
     \label{fig:sSFR_classification}
\end{figure}

\begin{figure}
    \centering
     \includegraphics[width=\columnwidth]{residual_sSFR_histogram.pdf}
     \caption{}
     \label{fig:Residual_sSFR}
\end{figure}

\begin{figure}
    \centering
     \includegraphics[width=\columnwidth]{galaxies_sfr.pdf}
     \caption{sSFR and morphologies of galaxies in \CG{} and our SDSS selection. 
     Quenched galaxies are represented at an arbitrary low value of $10^{-16.0}$ Gyr\(^{-1}\).}
     \label{fig:galaxies_sfr}
\end{figure}



\subsection{Morphology}
Morphologies are determined via the Galaxy Zoo citizen‐science decision tree, in which each galaxy image receives multiple 
independent volunteer classifications that are aggregated into debiaised vote fractions \cite{GalaxyZoo1}. Those fractions are provided in the 
\texttt{zooSpecz} table of the SDSS databasethrough the \texttt{p\_el\_debiased} and \texttt{p\_cs\_debiased} columns. We attribute to galaxies the
morphology ``elliptical'' if the \texttt{p\_el\_debiased} value is greater than 0.5, and ``spiral'' if the \texttt{p\_cs\_debiased} value is greater than 0.5. 
Other cases are considered as ``uncertain''. The morphologies found for each sample are shown in table \ref{tab:morphologies}.
Barnard two-sided exact tests were performed to compare the fraction of each morphology between \CG{} and each control sample. The p-values found are 
$1.6 \times 10^{-5}$ for ellipticals and $6.0 \times 10^{-1}$ for spirals versus \CB{} sample, 
 $7.4 \times 10^{-2}$  and $6.1 \times 10^{-1}$  versus \CC{} and  $2.8 \times 10^{-9}$  
 and $2.4 \times 10^{-2}$  versus \RG.
\begin{table}
    \centering
    \begin{tabular}{lccc}
        \toprule
            \textbf{Sample} & \textbf{  Elliptical  } & \textbf{  Spiral  } & \textbf{  Uncertain  } \\
            \CG &  144  (35.8 \%) &  158  (39.3 \%)&  100  (24.9 \%) \\
            \CB &  982  (25.4 \%) &  1465  (38.0 \%) &  1412  (36.6 \%) \\
            \CC &  1483  (31.5 \%) &  1912  (40.6 \%) &  1314  (27.9 \%) \\
            \RG &  37  (13.8 \%) &  129  (48.1 \%) &  102  (38.1 \%)\\
            \emph{SDSS selection} &  \emph{10091 (19.2 \%)} &  \emph{34715  (66.1 \%)} &  \emph{7725  (14.7 \%)}\\
         \midrule
        \bottomrule
    \end{tabular}
    \caption{Number of galaxies of each morphology for each sample.}
    \label{tab:morphologies}
\end{table}


% Once galaxies having uncertain morphology removed, \CG{} have  () of their galaxies classified as spiral and 
%  () elliptical,
% while the control sample has  () and  (), respectively. 
% The probability that those fractions come from 
% the same distribution is astimated as  from the Fisher two-sided exact test.

% % \begin{figure}
% %     \centering
% %      \includegraphics[width=\columnwidth]{.pdf}
% %      \caption{}
% %      \label{fig:}
% % \end{figure}

% \subsection{Dominance}
% We split groups into two categories: the dominated, for which the BGG has a luminosity fraction greater than the median of \RG.

\begin{appendix} %First appendix
\section*{Interpolation points for sSFR classification}

% \begin{table}[t]
% \centering
% \caption{Interpolation points for sSFR classification curve}
% \begin{tabular}{rr}
% \toprule
% $x$ & $y$ \\
% \midrule
% 
% 
% {{ "%.6g"|format(p.x) }} & {{ "%.6g"|format(p.y) }} \\
% 
% \\
% 
% \multicolumn{2}{c}{\rule{0pt}{0pt}}\\
% \multicolumn{2}{c}{No interpolation points available} \\
% 
% \bottomrule
% \end{tabular}
% \end{table}





\end{appendix}

\begin{acknowledgements}
We thank ... 
\end{acknowledgements}

\bibliographystyle{aa}
\bibliography{paper}

\end{document}